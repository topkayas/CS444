\documentclass[letterpaper,10pt,titlepage]{IEEEtran}

\usepackage{graphicx}                                        
\usepackage{amssymb}                                         
\usepackage{amsmath}                                         
\usepackage{amsthm}                                          

\usepackage{alltt}                                           
\usepackage{float}
\usepackage{color}
\usepackage{url}
\usepackage{upquote}

\usepackage{balance}
\usepackage[TABBOTCAP, tight]{subfigure}
\usepackage{enumitem}
\usepackage{pstricks, pst-node}
\usepackage[utf8]{inputenc}

\usepackage{geometry}
\geometry{textheight=8.5in, textwidth=6in}

%random comment

\newcommand{\cred}[1]{{\color{red}#1}}
\newcommand{\cblue}[1]{{\color{blue}#1}}

\usepackage{hyperref}
\usepackage{geometry}

\def\name{Sinan Topkaya}

%pull in the necessary preamble matter for pygments output
%\input{pygments.tex}

%% The following metadata will show up in the PDF properties
\hypersetup{
  colorlinks = true,
  urlcolor = black,
  pdfauthor = {\name},
  pdfkeywords = {cs444 ``OS2'' Rober Love Chapter 1 2},
  pdftitle = {CS 444 Week 1: Chapter 1 and 2 Summary},
  pdfsubject = {CS 444 Week 1},
  pdfpagemode = UseNone
}

\parindent = 0.0 in
\parskip = 0.2 in

\begin{document}
\begin{@twocolumnfalse}
	%\begin{titlepage}
		%\begin{center}
		%\\[1cm]
		%\textbf{Weekly Summaries}
		%\\[0.5cm]
		%by Sinan Topkaya
		%\end{center}
		%\\[1cm]
		%Abstract: This paper includes the chapter 1 and 2 summaries, from Linux Kernel Development book by Robert Love.
	%\end{titlepage}


	
\section*{Linux Kernel Development, Robert Love}
\subsection*{Chapter 1 and 2 Summary}

Robert Love, author of Linux Kernel Development (September, 2003) book, assert Unix operating system as one of the most powerful and elegant systems in existence by claiming it to being simple to use by giving cited examples of how much it have improved throughout the time. The author proves his claim by comparing different versions of Linux Kernel, explaining in detail the features it currently supports. The author\textquotesingle s purpose in the first 2 chapters of his book, is to get the reader to learn the history of the system, its features and how to build the Kernel in order to start using the system. The intended audience for these chapters are people that would like to learn more about the system and start benefiting from its features by following the author\textquotesingle s in depth step-by-step explanation of how to install and build the system, he establishes mentor-student relationship with his audience by suggesting that kernel is not something to fear and that if the reader continues reading the book, they won’t have any problem understanding the rules and the stakes, managing their kernel system.
\end{@twocolumnfalse}
\end{document}

