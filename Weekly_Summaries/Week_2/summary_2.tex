\documentclass[letterpaper, onecolumn, draftclsnofoot,10pt,titlepage]{IEEEtran}

\usepackage{graphicx}                                        
\usepackage{amssymb}                                         
\usepackage{amsmath}                                         
\usepackage{amsthm}                                          

\usepackage{alltt}                                           
\usepackage{float}
\usepackage{color}
\usepackage{url}
\usepackage{upquote}

\usepackage{balance}
\usepackage[TABBOTCAP, tight]{subfigure}
\usepackage{enumitem}
\usepackage{pstricks, pst-node}

\usepackage{geometry}
\geometry{textheight=8.5in, textwidth=6in}

%random comment

\newcommand{\cred}[1]{{\color{red}#1}}
\newcommand{\cblue}[1]{{\color{blue}#1}}

\usepackage{hyperref}
\usepackage{geometry}

\def\name{Sinan Topkaya}

%pull in the necessary preamble matter for pygments output
%\input{pygments.tex}

%% The following metadata will show up in the PDF properties
\hypersetup{
  colorlinks = true,
  urlcolor = black,
  pdfauthor = {\name},
  pdfkeywords = {cs444 ``OS2'' Rober Love Chapter 3 4},
  pdftitle = {CS 444 Week 2: Chapter 3 and 4 Summary},
  pdfsubject = {CS 444 Week 2},
  pdfpagemode = UseNone
}

\parindent = 0.0 in
\parskip = 0.2 in

\begin{document}

\section*{Linux Kernel Development, Robert Love}
\subsection*{Chapter 3 and 4 Summary}

Robert Love, author of Linux Kernel Development (September, 2003) book, asserts that Linux Kernel offers the optimal solution for most use cases to achieve process scheduling and process management for hard algorithms. The author proves his claim by comparing Linux Kernel to many other operating systems, and by introducing valuable functions and their purposes that makes Linux Kernel optimal to use. The author\textquotesingle s purpose in chapter 3 and 4 of his book, is to introduce the core operating system abstraction of the process, the relationship between process and threads and also the process scheduler to the readers in order for them to properly run processes, without speed problems and scalability issues. The intended audience for these chapters are people that would like to learn more about process management, theory behind process scheduling and the specific implementation, algorithms, and interfaces using Linux Kernel by following the author\textquotesingle s in depth step-by-step explanation and flow charts.
\end{document}

