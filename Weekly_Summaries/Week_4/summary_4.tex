\documentclass[letterpaper,10pt,draftclsnofoot,onecolumn]{IEEEtran}

\usepackage{graphicx}                                        
\usepackage{amssymb}                                         
\usepackage{amsmath}                                         
\usepackage{amsthm}                                          

\usepackage{alltt}                                           
\usepackage{float}
\usepackage{color}
\usepackage{url}
\usepackage{upquote}

\usepackage{balance}
\usepackage[TABBOTCAP, tight]{subfigure}
\usepackage{enumitem}
\usepackage{pstricks, pst-node}
\usepackage[utf8]{inputenc}

\usepackage{geometry}
\geometry{textheight=8.5in, textwidth=6in}

%random comment

\newcommand{\cred}[1]{{\color{red}#1}}
\newcommand{\cblue}[1]{{\color{blue}#1}}

\usepackage{hyperref}
\usepackage{geometry}

\def\name{Sinan Topkaya}

%pull in the necessary preamble matter for pygments output
%\input{pygments.tex}

%% The following metadata will show up in the PDF properties
\hypersetup{
  colorlinks = true,
  urlcolor = black,
  pdfauthor = {\name},
  pdfkeywords = {cs444 ``OS2'' Rober Love Chapter 6 7},
  pdftitle = {CS 444 Week 4: Chapter 6 and 7 Summary},
  pdfsubject = {CS 444 Week 4},
  pdfpagemode = UseNone
}

\parindent = 0.0 in
\parskip = 0.2 in

\begin{document}

\begin{titlepage}
	
	\begin{center}
	\bigbreak
	\textbf{Weekly Summaries - Week 4}
	\bigbreak
	by Sinan Topkaya
	\smallbreak
	CS 444 - Spring 2016
	\end{center}
\end{titlepage}
	
\section*{Linux Kernel Development, Robert Love}
\subsection*{Chapter 6 and 7 Summary - Kernel Data Structures and Inturrupts and Handlers}

Robert Love, author of Linux Kernel Development (September, 2003) book, assert that Linux Kernel operating system offers generic built-in easy to use data structures that encourage code resuse , he also assert that Kernel supports most modern hardwares by allowing to use interrupts. The author proves his claim by introducing data structures and how to use them in chapter 6 of his book, showing simple codes and introducing readers the most well-known notations and their importance, and on chapter 7 of his book he introduces the importance of interrupt handlers and explains the user how to control the interrupts. Overall chapters 6 and 7 of his book the author teaches his readers different data-structures, how use to use them in a code and also teaches his readers how Kernel implements interrupt handlers and how the user can control them. The intended audience for these chapters are people that would like to learn more Interrupts and Interrupt Handlers and how Linux Kernel manages the hardware connection to the machine without impacting the machine’s overall performance and also users that would like to learn how to use data structures to implement everything in the operating system from the process scheduler to device drivers.

\end{document}

