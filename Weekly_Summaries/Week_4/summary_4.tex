\documentclass[letterpaper,10pt,draftclsnofoot,onecolumn]{IEEEtran}

\usepackage{graphicx}                                        
\usepackage{amssymb}                                         
\usepackage{amsmath}                                         
\usepackage{amsthm}                                          

\usepackage{alltt}                                           
\usepackage{float}
\usepackage{color}
\usepackage{url}
\usepackage{upquote}

\usepackage{balance}
\usepackage[TABBOTCAP, tight]{subfigure}
\usepackage{enumitem}
\usepackage{pstricks, pst-node}
\usepackage[utf8]{inputenc}

\usepackage{geometry}
\geometry{textheight=8.5in, textwidth=6in}

%random comment

\newcommand{\cred}[1]{{\color{red}#1}}
\newcommand{\cblue}[1]{{\color{blue}#1}}

\usepackage{hyperref}
\usepackage{geometry}

\def\name{Sinan Topkaya}

%pull in the necessary preamble matter for pygments output
%\input{pygments.tex}

%% The following metadata will show up in the PDF properties
\hypersetup{
  colorlinks = true,
  urlcolor = black,
  pdfauthor = {\name},
  pdfkeywords = {cs444 ``OS2'' Rober Love Chapter 6 7},
  pdftitle = {CS 444 Week 4: Chapter 6 and 7 Summary},
  pdfsubject = {CS 444 Week 4},
  pdfpagemode = UseNone
}

\parindent = 0.0 in
\parskip = 0.2 in

\begin{document}

	%\begin{titlepage}
		%\begin{center}
		%\\[1cm]
		%\textbf{Weekly Summaries}
		%\\[0.5cm]
		%by Sinan Topkaya
		%\end{center}
		%\\[1cm]
		%Abstract: This paper includes the chapter 1 and 2 summaries, from Linux Kernel Development book by Robert Love.
	%\end{titlepage}


	
\section*{Linux Kernel Development, Robert Love}
\subsection*{Chapter 6 and 7 Summary - Kernel Data Structures and Inturrupts and Handlers}

Robert Love, author of Linux Kernel Development (September, 2003) book, assert that Linux Kernel operating system has one of the most developed and simple I/O Layer compared to other operating systems. The author proves his claim by showing his readers different data structures used by the block I/O in detail and with examples, he also proves his claim by showing the old versions of the Linux Kernel and how they used to handle I/O layers. The author\textquotesingle s purpose in chapter 14 of his book, is to get the reader to learn more about I/O layers, the data structures they use, relationship between each data structure and different I/O schedulers, he explains it by giving a lot of examples and comparable information, so the reader knows what to use and when. The intended audience for these chapters are people that would like to learn more about data structures used by block I/O layer, I/O requests and schedulers, by following the author\textquotesingle s in depth step-by-step explanation of how to use them and when to use which one and why.
\end{document}

