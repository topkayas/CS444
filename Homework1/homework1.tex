\documentclass[letterpaper,10pt,draftclsnofoot,onecolumn]{IEEEtran}

\usepackage{graphicx}                                        
\usepackage{amssymb}                                         
\usepackage{amsmath}                                         
\usepackage{amsthm}                                          

\usepackage{alltt}                                           
\usepackage{float}
\usepackage{color}
\usepackage{url}
\usepackage{upquote}
\usepackage{array}
\usepackage{balance}
\usepackage[TABBOTCAP, tight]{subfigure}
\usepackage{enumitem}
\usepackage{pstricks, pst-node}
\usepackage[utf8]{inputenc}

\usepackage{geometry}
\geometry{textheight=8.5in, textwidth=6in}

%random comment

\newcommand{\cred}[1]{{\color{red}#1}}
\newcommand{\cblue}[1]{{\color{blue}#1}}

\usepackage{hyperref}
\usepackage{geometry}

\def\name{Sinan Topkaya}

%pull in the necessary preamble matter for pygments output
%\input{pygments.tex}

%% The following metadata will show up in the PDF properties
\hypersetup{
  colorlinks = true,
  urlcolor = black,
  pdfauthor = {\name},
  pdfkeywords = {cs444 ``OS2'' Homework 1},
  pdftitle = {CS 444 Homework 1: Project and Conquerency},
  pdfsubject = {CS 444},
  pdfpagemode = UseNone
}

\parindent = 0.0 in
\parskip = 0.2 in

\begin{document}

	\include(gitlog.tex)
	\begin{titlepage}
		
		\begin{center}
		\bigbreak	
		\textbf{Project 1 and Conquerency 1}
		\bigbreak
		by Sinan Topkaya
		\smallbreak
		CS 444 - Spring 2016
		\end{center}
		\vfill
		
		Abstract: This paper includes information on how to build and work on kernel plus information on choices I made making a multi threded consumer producer program.
		
	\end{titlepage}

\section*{Write-up}
\subsection*{Project 1}

\subsubsection*{Log of commands for Project1} 
\begin{itemize}
	\item git init  
	\item git clone git://gityoctoproject.org/linux-yocto-3.14 
	\item git checkout -b v3.14.26 
	\item git tag -l 
	\item cp .config cp sb-sdk-qemux.ex3 
	\item source /scratch/opt/environment-setup-i586-linux 
	\item make -j4 all 
	\item qemu-system-i386 -gdb tcp::5630 -S -nographic -kernel linux-yocto-3.14/arch/x86/boot/bzImage -drive file=core-image-lsb-sdk-qemux86.ext3,if=virtio -enable-kvm -net none -usb -localtime --no-reboot --append "root=/dev/vda rw console=ttyS0 debug" 
	\item opened another putty session 
	\item gdb 
	\item target remote localhost:5630 
	\item c 
	\item uname -r 
\end{itemize}

\subsubsection*{Qemu Command and its flags} 
\begin{itemize}
	\item The code: qemu-system-i386 -gdb tcp::5630 -S -nographic -kernel linux-yocto-3.14/arch/x86/boot/bzImage -drive file=core-image-lsb-sdk-qemux86.ext3,if=virtio -enable-kvm -net none -usb -localtime --no-reboot --append "root=/dev/vda rw console=ttyS0 debug" 
	\item -gdb tcp::5630: This flag allows us to debug the program using gdb, we have to connect to localhost 5630 to see what is exactly going on with the running program 
	\item -S: This option makes qemu launch wait till gdb connection. 
	\item -nongraphic: Disables grahical putput so that QEMU is a simple command line application.
	\item -kernel: This option is needed to provide the Linux kernel image.
	\item -drive: defines the drive. 
	\item -if=virtio: specifies the controller's PCI. 
	\item -enable-kvm: Enables KVM full virtualization support. 
	\item -net none: Indicates that no network devices should be configured. 
	\item -usb: Enables USB drivers.
	\item -localtime: Set RTC start at local time. 
	\item --no-reboot: Exit instead of rebooting. 
	\item --append: is used to give the kernel command line argument. 
\end{itemize}

\subsection*{Conqurency 1}

\subsubsection*{Reflection}
	1. I think the main point was to learn how to use program multi-threaded programs, this assignment refreashed my memory working with them.\\
	2. Since I did not remember too much, I read and watched few videos on producer-consumer problems. Also I approached the assignment piece by piece I created threads then worked on producer first, once I was sure the thread was doing what it was supposed to do, I continued working on the consumer.\\
	3. Approaching the problem piece by piece made it much easier. As mentioned above I made sure the producer thread worked right before I started working on the consumer, I have also ran the program multiple times and waited if I am generating correct random numbers or not.\\
	4. Well I could say I have learned alot about threads. It was actually a really fun assignment to learn more about threads and how to handle them, also thinking about error it might have was really interresting like blocking signals.\\

\subsection*{git log}
\begin{tabular}{l l l}\textbf{Detail} & \textbf{Author} & \textbf{Description}\\\hline
\href{git@github.com:topkayas/CS444/commit/e0f6bfaafb812f0756fcb6deb6758c3fe1bf53b0}{e0f6bfa} & topkayas & first commit\\\hline
\href{git@github.com:topkayas/CS444/commit/6443ef797e6589433f3e981c7adaa47f24fd53ba}{6443ef7} & topkayas & First Commit Kernel v3.14.26\\\hline
\href{git@github.com:topkayas/CS444/commit/44dc181c2681efe404325a17c74427d494993060}{44dc181} & topkayas & Summary Week 1\\\hline
\href{git@github.com:topkayas/CS444/commit/040e0b0809d4b0b58787aafb1f8cf2cd4fe2962b}{040e0b0} & topkayas & updates\\\hline
\href{git@github.com:topkayas/CS444/commit/b489bd05b7672cd26e9542f995aac2a7e6d11635}{b489bd0} & topkayas & update\\\hline
\href{git@github.com:topkayas/CS444/commit/12818e6c4169cd06beaf130670f79b7111f5de41}{12818e6} & topkayas & update\\\hline
\href{git@github.com:topkayas/CS444/commit/d30ab815100d1cdb8c11f731b60f8d8bc801a310}{d30ab81} & topkayas & week 2 summary\\\hline
\href{git@github.com:topkayas/CS444/commit/18c007a6079a2a1c950f3b8abd160d71373ecc84}{18c007a} & topkayas & IEEEtran\\\hline
\href{git@github.com:topkayas/CS444/commit/9c9e7bf414a403688cf51d2e12bfb4cd81a9865c}{9c9e7bf} & topkayas & summaries done\\\hline
\href{git@github.com:topkayas/CS444/commit/f4d795117875c265bab1dbd47ea4acda59f437f1}{f4d7951} & topkayas & title page\\\hline
\href{git@github.com:topkayas/CS444/commit/da69ee05be6968f69c8222b99384447b3d0ecc21}{da69ee0} & topkayas & summary\\\hline
\href{git@github.com:topkayas/CS444/commit/47b8240462ba004452a13729ffadf92ccaa6f105}{47b8240} & topkayas & last changes\\\hline
\href{git@github.com:topkayas/CS444/commit/74dbc03c78cb2e617df6e9048f06bd10cb2c8e9b}{74dbc03} & topkayas & 1 column\\\hline
\href{git@github.com:topkayas/CS444/commit/902be284317fd5be08fec57a7fd557e6b3356dde}{902be28} & topkayas & one column\\\hline
\href{git@github.com:topkayas/CS444/commit/eaba13e546007ced6fde62d4111d2ba338b9287d}{eaba13e} & topkayas & changes\\\hline
\href{git@github.com:topkayas/CS444/commit/8937dde711a688881683e2152fe987dca0021c3a}{8937dde} & topkayas & changes\\\hline
\href{git@github.com:topkayas/CS444/commit/cbbc8ff7c82f89baa2d096fe8a62432bf2d8002a}{cbbc8ff} & topkayas & starting with the pdf file\\\hline
\href{git@github.com:topkayas/CS444/commit/77d005ba46dff628ae08cbf01501ac260fa9bae4}{77d005b} & topkayas & pdf\\\hline
\href{git@github.com:topkayas/CS444/commit/99459b04d20c397edfe76334ee23804c014fa7fd}{99459b0} & topkayas & change pdf\\\hline
\href{git@github.com:topkayas/CS444/commit/c1e529725300d2017c355c06db4a17df5557cbb5}{c1e5297} & topkayas & update\\\hline
\href{git@github.com:topkayas/CS444/commit/3eefb8fe8b6feb386fb728920dc22866fe264185}{3eefb8f} & topkayas & update\\\hline
\href{git@github.com:topkayas/CS444/commit/ff63e14ca05eff15c6eabeb6004dfbeadb56d630}{ff63e14} & topkayas & update\\\hline
\href{git@github.com:topkayas/CS444/commit/9003cf91fc096990444cbdebb32a0b81b6eccc68}{9003cf9} & topkayas & update\\\hline
\href{git@github.com:topkayas/CS444/commit/91630e465af29a2114228f046547399cb34dc840}{91630e4} & topkayas & update latex\\\hline
\href{git@github.com:topkayas/CS444/commit/ce0904949a518e35ceb040fc1c678832a315bc9c}{ce09049} & topkayas & update latex\\\hline
\href{git@github.com:topkayas/CS444/commit/dcf140f4c18d0aaf99c6cb44ccfbd28eb8c82e65}{dcf140f} & topkayas & update latex\\\hline
\href{git@github.com:topkayas/CS444/commit/c044b5690fcbcef470fbb0be53975ae3b1f1e3c8}{c044b56} & topkayas & homework finished\\\hline\end{tabular}

\subsection*{work log}
\begin{center}
\begin{tabular}{ |m{2cm}|m{5cm}| }
\hline
When & What \\
4/4/2016 & Started with Project 1, connected to the os-class and ran the already given bzImage, got familiar with git and build my own version control system.\\
4/5/2016 & Built kernel using git because last time I could not. Started with latex, but end up removing it.\\
4/6/2016 & Everything finally worked well with qemu and Project 1 is completely finished.\\
4/7/2016 & Forgot to commit changes. I started with implementing threads and tested if they were succesfully created\\
4/8/2016 & Created the producer thread and function, it works fine and started working on the cpnsumer\\
4/9/2016 & I am having troubles deleting and I will start working on the consumer\\
4/10/2016 & Worked on Weekly Summaries\\
4/11/2016 & Working one th consumer thread, the program runs well but needs few fixes. \\
\hline
\end{tabular}
\end{center}
\end{document}
